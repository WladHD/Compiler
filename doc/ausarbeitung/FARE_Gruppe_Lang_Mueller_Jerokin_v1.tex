\documentclass[a4paper,11pt,numbers=noenddot]{article}

\usepackage[main=ngerman, english]{babel}
%\usepackage{natbib}
%\bibliographystyle{plainnat}
%\setcitestyle{square,aysep={},yysep={;}}
%\setcitestyle{authoryear,square,semicolon} 
%\setcitestyle{numbers,square}

\usepackage[latin1]{inputenc}
\usepackage[babel, german=quotes]{csquotes} % einfache Handhabung von quotations
\usepackage[backend=bibtex8,style=authoryear]{biblatex} %biblatex mit biber laden
\ExecuteBibliographyOptions{
	sorting=nyt, %Sortierung Autor, Titel, Jahr
	bibwarn=true, %Probleme mit den Daten, die Backend betreffen anzeigen
	isbn=false, %keine isbn anzeigen
	url=true, %keine url anzeigen
	maxcitenames=2,
	maxbibnames=99
}
\renewcommand*{\nameyeardelim}{\addcomma\space}
\addbibresource{res/literaturBibLatex.bib} %Bibliographiedateien laden

\setlength{\parindent}{1em}
\linespread{1.2}
\usepackage[T1]{fontenc}
\usepackage{hyperref}

\hypersetup{
	colorlinks=true,
	linkcolor=black,
	filecolor=black,      
	urlcolor=black,
	citecolor=black,
	pdftitle={FARE Compiler} % PDFTITLE
}


\usepackage{geometry}
\usepackage{textpos} 
\usepackage[table]{xcolor}
\definecolor{lightgray}{gray}{0.9}
\usepackage{graphicx}
\usepackage{float}
\usepackage{csquotes}
\usepackage{ifthen}
\usepackage{booktabs}
\usepackage{tabu}
\usepackage[inline]{enumitem}
\usepackage{enumitem}
\usepackage{calc}
\usepackage{pifont}
\usepackage{colortbl}
\usepackage{longtable}
\usepackage{wrapfig}
\usepackage{listings}
\usepackage{subfig}
\usepackage{ragged2e}
\usepackage{xcolor}

\newcommand\signature[2]{% Name; Department
	\noindent\begin{minipage}{6.5cm}
		\noindent\vspace{3cm}\par
		\noindent\rule{6.5cm}{1pt}\par
		\noindent\textbf{#1}\par
		\noindent#2%
\end{minipage}}

\geometry{a4paper, left=45mm, right=15mm, top=30mm, bottom=25mm, headsep=1cm, headheight=0cm}
\definecolor{accent}{rgb}{0.41, 0.6, 0.76}

\usepackage{fancyhdr}
\usepackage[nottoc]{tocbibind}

\renewcommand{\headrulewidth}{0pt}
\renewcommand{\footrulewidth}{0pt}

\fancyhf{}
\fancyhead[C]{\Roman{page}}
\pagestyle{fancy}

\fancypagestyle{fancy}{}

\definecolor{lightgray}{rgb}{.9,.9,.9}
\definecolor{darkgray}{rgb}{.4,.4,.4}
\definecolor{purple}{rgb}{0.65, 0.12, 0.82}

\lstdefinelanguage{JavaScript}{
	keywords={typeof, new, true, false, catch, function, return, null, catch, switch, var, if, in, while, do, else, case, break},
	keywordstyle=\color{blue}\bfseries,
	ndkeywords={class, export, boolean, throw, implements, import, this},
	ndkeywordstyle=\color{darkgray}\bfseries,
	identifierstyle=\color{black},
	sensitive=false,
	comment=[l]{//},
	morecomment=[s]{/*}{*/},
	commentstyle=\color{purple}\ttfamily,
	stringstyle=\color{black}\ttfamily,
	morestring=[b]',
	morestring=[b]"
}

\lstset{
	language=JavaScript,
	backgroundcolor=\color{lightgray},
	extendedchars=true,
	basicstyle=\footnotesize\ttfamily,
	showstringspaces=false,
	showspaces=false,
	numbers=left,
	numberstyle=\footnotesize,
	numbersep=9pt,
	tabsize=2,
	breaklines=true,
	showtabs=false,
	captionpos=b
}

\newenvironment{changemargin}[2]{%
	\begin{list}{}{%
			\setlength{\topsep}{0pt}%
			\setlength{\leftmargin}{#1}%
			\setlength{\rightmargin}{#2}%
			\setlength{\listparindent}{\parindent}%
			\setlength{\itemindent}{\parindent}%
			\setlength{\parsep}{\parskip}%
		}%
		\item[]}{\end{list}}

\lstdefinelanguage{none}{
	identifierstyle=
}

\colorlet{punct}{red!60!black}
\definecolor{background}{HTML}{EEEEEE}
\definecolor{delim}{RGB}{20,105,176}
\colorlet{numb}{magenta!60!black}

\lstdefinelanguage{json}{
	basicstyle=\footnotesize\ttfamily,
	numbers=left,
	numberstyle=\footnotesize,
	stepnumber=1,
	numbersep=8pt,
	showstringspaces=false,
	breaklines=true,
	frame=lines,
	backgroundcolor=\color{background},
	literate=
	*{0}{{{\color{numb}0}}}{1}
	{1}{{{\color{numb}1}}}{1}
	{2}{{{\color{numb}2}}}{1}
	{3}{{{\color{numb}3}}}{1}
	{4}{{{\color{numb}4}}}{1}
	{5}{{{\color{numb}5}}}{1}
	{6}{{{\color{numb}6}}}{1}
	{7}{{{\color{numb}7}}}{1}
	{8}{{{\color{numb}8}}}{1}
	{9}{{{\color{numb}9}}}{1}
	{:}{{{\color{punct}{:}}}}{1}
	{,}{{{\color{punct}{,}}}}{1}
	{\{}{{{\color{delim}{\{}}}}{1}
	{\}}{{{\color{delim}{\}}}}}{1}
	{[}{{{\color{delim}{[}}}}{1}
	{]}{{{\color{delim}{]}}}}{1},
}


\usepackage[printonlyused]{acronym}

\makeatletter
\renewcommand*\AC@acs[1]{%
	\expandafter\AC@get\csname fn@#1\endcsname\@firstoftwo{#1}}
\makeatother

%biblatex patch, working now
%\linespread{1.25}
%\makeatletter
%\def\blx@err@patch#1{}
%\makeatother

\usepackage{setspace}
\usepackage[automake]{glossaries}
\glsdisablehyper

\newglossary[nlg]{nu}{not}{ntn}{Not Used}

\newglossaryentry{Widget}
{
	type=nu,
	name=Widget,
	description={Ein Widget ist innerhalb des Dashboards ein Element, welches Informationen anzeigt und interaktiv verwendet werden kann},
	plural=Widgets,
	%nonumberlist
}

\makeglossaries

\colorlet{punct}{red!60!black}
\definecolor{background}{HTML}{EEEEEE}
\definecolor{delim}{RGB}{20,105,176}
\colorlet{numb}{magenta!60!black}
\definecolor{accent}{rgb}{0.41, 0.6, 0.76}

\begin{document}
	\begin{titlepage}
		\begin{textblock}{6.5}(-2.5,-3)
			\begin{color}{accent}
				\rule{6.6cm}{33cm}    
			\end{color}
		\end{textblock}
		\begin{textblock}{6.5}(-2,-1)
			{\large \textsf{Ausarbeitung}}
		\end{textblock}
		
		\doublespacing
		\begin{textblock}{8.2}(3.1,1)
			{
				\noindent \LARGE 
				\textsf{\textbf{Entwicklung eines Compilers f�r die Sprache FARE zur Zielsprache Java \\[0.3cm]
			} }}
		\end{textblock}
		\onehalfspacing
		
		\begin{textblock}{8.2}(3.1,3.5)
			{\noindent \large
				\textsf{\textbf{Development of a compiler for the language FARE\\ to the target language Java  \\[0.3cm]
			} }}
		\end{textblock}
		
		\begin{textblock}{6}(3.1,5.5)
			\noindent
			\textsf{An der Fachhochschule Dortmund\\
				im Fachbereich Informatik\\
				Studiengang Medizinische Informatik Master\\
				im Modul Formale Sprachen und Compilerbau\\
				erstellte Ausarbeitung eines FARE-Compilers
			}
		\end{textblock}
		
		
		
		
		\begin{textblock}{6}(-2,7.5)\noindent
			\textsf{von \\Johannes Lang \\
				Matr.-Nr. 7217450 \\ [0.2cm]
				Henning M�ller \\
				Matr.-Nr. 7105852 \\ [0.2cm]
				Wladislaw Jerokin \\
				Matr.-Nr. 7205290 \\ [1cm]
				Betreuung durch: \\
				Prof. Dr. Robert Rettinger \\ [1cm]
				%In enger Zusammenarbeit mit: \\
				%Dr. Georg Lodde \\ [1cm]
				Dortmund, \today
			}
		\end{textblock}
		
		%	\begin{textblock}{6.5}(-1,10.8)
			%		\noindent
			%			\textsf{An der Fachhochschule Dortmund\\
				%			im Fachbereich Informatik\\
				%			Studiengang Medizinische Informatik\\
				%			erstellte Projektarbeit f�r das \\
				%			Modul Wissenschaftliches Arbeiten
				%		}
			%	\end{textblock}
		
	\end{titlepage}
	
	\onehalfspacing
	\setlength\arrayrulewidth{1.1pt}
	\newpage
	\tableofcontents
	
	\newpage
	\setcounter{page}{1}
	\fancyhf{}
	\fancyhead[C]{\thepage}
	\pagestyle{fancy}
	\fancypagestyle{fancy}{}
	
	\section{Einleitung}
	Grundlegend definiert sich ein Compiler als Programm, welches einen gegebenen Quellcode zu Maschinencode, Bytecode oder einer anderen Programmiersprache �bersetzen kann \parencite[vgl.][]{RobertSheldon.2023}.
	Die Entwicklung eines solchen Compilers ist eine komplexe Aufgabe, die aus mehreren Teilgebieten besteht.
	In dieser Ausarbeitung werden folgende Teilgebiete behandelt:
	
	\begin{longtable}{|c|c|}
		\toprule{}
		Num & Name \\ \midrule  
		
		1 & Lexikalische Analyse \\
		
		2 & Syntaxanalyse  \\
		
		3 & Semantische Analyse  \\
		
		4 & Fehlerbehandlung  \\
		
		5 & Codeerzeugung  \\ \midrule
		
		\caption{Behandelte f�nf Teilbereiche des Compilerbaus}
		\label{table:bereiche}
	\end{longtable}
	
	\noindent
	Die in dieser Ausarbeitung behandelte Aufgabe besteht darin, die Bereiche in Tabelle \ref{table:bereiche} f�r eine Scriptsprache f�r den Umgang mit Dateien und Pfaden zu entwickeln.
	Folgend werden die herausgearbeiteten Token, Grammatik und Semantik in jeweils eigenen Kapiteln beschrieben.
	
	\subsection{Herangehensweise}
	\label{sec:Herangehensweise}
	Der Compiler wird in Java geschrieben und benutzt die Bibliotheken JavaCC und die JavaCC-interne JJTree.
	Somit wird die Lexikalische Analyse und die Syntaxanalyse durch die definierte Grammatik durchgef�hrt.
	Um dies kurz auszuf�hren wird mit JavaCC gepr�ft, ob ein gegebener Quellcode zur definierten Sprache und Grammatik zugeh�rt (Lexikalische Analyse).
	Durch erweiterte Annotation der Grammatik wird folgend der AST durch JJTree generiert (Syntaxanalyse).
	Falls keine Fehler vorliegen, wird der AST zur Semantischen Analyse und Fehlerbehandlung �bergeben, wo auf semantische Korrektheit gepr�ft wird.
	Abschlie�end wird bei korrekter Semantik der AST zur Codeerzeugung �bergeben, bei welcher vorerst Java Source Code erzeugt und danach compiliert wird.
	
	
	
	
	
	
	\newpage
	\section{Token}
	\label{sec:Token}
	\begin{lstlisting}[caption={Definierte JavaCC Token}, label=lst:token]
// de/fh/javacc/Grammar1.jjt, Zeile 20 - 54
<TypeMap: "Map">
| <TypeSet: "Set">
| <TypeSpecifiers: "int" | "char" | "String" | "boolean"| "Files" >
| <LBRACKET : "[">
| <RBRACKET : "]">
| <TypePath: "Path" >
| <IF: "if">
| <ELSE: "else">
| ";" |  ","
| <GBracketOpen: "{">
| <GBracketClose: "}">
| <TypeVoid: "void">
| <While: "while">
| <For: "for">
| <SimpleEquals: "="| "+=" |"-=" |"*="| "/=" | "%=" >
| <Point: ".">
| <RoundBracketOpen: "(">
| <RoundBracketClose: ")">
| <BooleanLiteral: "true" | "false" >
| <LessThan: "<">
| <GreaterThan: ">">
| <BinVergleich : ">="  | "<=" | "==" | "!=" >
| <BinJunktor : "||" | "&&" >
| <OpSum : "+" | "-" >
| <OpUnaer: "!">
| <OpIncrement: "++" | "--">
| <OpProd : "*" | "%">
| <FWD: "/">
| <Return : "return">
| <Identifier : ["A"-"Z", "a"-"z", "_"](["A"-"Z", "a"-"z", "_", "0" - "9"])* >
| <IntegerLiteral : "0" | (["1"-"9"] (["0"-"9"])*) >
| <StringLiteral: "\""(~["\"","\n"])*"\"" >
| <CharLiteral:"'" (~["'","\n"]) "'" >
| <PathLiteral:".."|  ["A"-"Z"] ":">
	\end{lstlisting}
	
	\noindent
	Insgesamt werden 34 verschiedene Token in Listing \ref{lst:token} definiert. Wenn nicht anders genannt, wird im Folgenden bei Zeilenangaben Listing \ref{lst:token} referenziert.
	Die akzeptierten Variablentypen der Programmiersprache sind \textit{Map, Set, int, char, String, boolean, Files, Path} (vgl. Z. 2 - 4, 7).
	Methoden erweitern diese um den Typen \textit{void} (vgl. Z. 13).
	Zul�ssige Ausdr�cke sind \textit{if, else, while, for} (vgl. Z. 8, 9, 14, 15).
	Zuweisungen und Operationen werden nach dem Java Operator Vorrang eingeteilt \parencite[vgl.][]{javatpoint.2023}.
	Diese werden in Zeilen 5, 6 sowie 16 bis 29 definiert.
	Als grundlegende Token werden \textit{Identifier} und Literale f�r \textit{Integer, String, Char und Path} definiert.
	Bei \textit{PathLiteral} handelt es sich um eine Sonderform. Der Pfad an sich wird durch grammatische Regeln definiert und ist in Kapitel \ref{sec:Grammatik} beschrieben.
	
	\section{Grammatik}
	\label{sec:Grammatik}
	Wie in Kapitel \ref{sec:Herangehensweise} genannt, wird JavaCC und JJTree genutzt. Um die erweiterte Grammatik zu demonstrieren, wird im Folgenden ein Beispiel angebracht.
	
		\begin{lstlisting}[caption={Beispiel der erweiterten Grammatik f�r Nutzung von JavaCC und JJTree anhand der Regel atom()}, label=lst:atom]
// de/fh/javacc/Grammar1.jjt, Zeile 436 - 449
void atom() #void :
{
	Token ident = null;
} {
	LOOKAHEAD(2)
	methodOrVariableCaller()
	| ident=<Identifier> {jjtThis.value = ident.image; } #ATOM_VARIABLE
	| ident=<IntegerLiteral> {jjtThis.value = ident.image; } #ATOM_INT
	| ident=<BooleanLiteral> {jjtThis.value = ident.image; } #ATOM_BOL
	| ident = <StringLiteral> {jjtThis.value = ident.image; } #ATOM_STRING
	| ident = <CharLiteral> {jjtThis.value = ident.image; } #ATOM_CHAR
	| (LOOKAHEAD(3) array()| atomset()| atommap())
	| <RoundBracketOpen> junktoren() <RoundBracketClose>
}
	\end{lstlisting}
	
	\noindent
	�hnlich zum reinen JavaCC k�nnen Regeln f�r die Grammatik definiert werden (vgl. Listing \ref{lst:atom}, Z. 7 - 16).
	Die Erweiterung liegt in der Annotation f�r die Erstellung des AST, welche die Werte \textit{void, ATOM\_VARIABLE, ATOM\_INT, ATOM\_BOL, ATOM\_STRING, ATOM\_CHAR} annehmen kann. Im Sachkontext wird im Normalfall keine Node im AST angelegt (vgl. Listing \ref{lst:atom}, Z. 2).
	Sollte jedoch eine Regel in den Zeilen 8 - 12 angewandt werden, so wird ein entsprechend benannter Knoten zum AST mit dem Inhalt des Tokens hinzugef�gt (vgl. Listing \ref{lst:atom}, Z. 8 - 12). Nach diesem Prinzip wird f�r alle Regeln an sinnvollen Stellen ein Knoten zum AST hinzugef�gt. Diese Kombination von JavaCC und JJTree erm�glicht die simultane Lexikalische Analyse und Syntaxanalyse.
	
	Nach der Anf�hrung eines Beispiels wird folgend grob die Grammatik beschrieben. Insgesamt liegen 51 grammatische Regeln vor.
\begin{lstlisting}[caption={Wurzel des AST sowie erste Regel der Grammatik}, label=lst:rootNode]
// de/fh/javacc/Grammar1.jjt, Zeile 62 - 70
SimpleNode compilationUnit() #PROGRAM :
{
	boolean first = true;
	SimpleNode result = null;
} {(
	(LOOKAHEAD(2) stmnt() | decl() | (methodOrVariableCaller() ";")) { if (first) result = jjtThis; else result.jjtAddChild(jjtThis, result.jjtGetNumChildren()); }
	)*
	{ return result; }
}
\end{lstlisting}	
	
	\noindent
	Da die Ausf�hrung der ersten Regel einen AST als R�ckgabewert liefert, stellt die Regel in Listing \ref{lst:rootNode} eine Sonderform dar.
	Wie erkannt werden kann, wird in jedem Fall ein neuer Knoten mit dem Namen \textit{PROGRAM} erzeugt.
	Unter diesem Knoten k�nnen nun \textit{stmnt(), decl()} und Methodenaufrufe vorliegen.
	Fortf�hrend wird beispielhaft die Regel zur Erzeugung des Literals \path{Path} angebracht.
	\begin{lstlisting}[caption={Wurzel des AST sowie erste Regel der Grammatik}, label=lst:rootNode]
// de/fh/javacc/Grammar1.jjt, Zeile 314 - 325
void path()  #ATOM_PATHELEMENT:
{Token t;
	Token t1 = null;
}

{
	
	(
	t = <PathLiteral> {jjtThis.value = t.image;} |
	t = <Point> {jjtThis.value = t.image;}|
	t = <Identifier> ["." t1= <Identifier>]{jjtThis.value = (t.image +  (t1 != null ? "." + t1.image : ""));}) [("/" | "\\") [Path()]]
}
	\end{lstlisting}	
	
	
	
	
	
	
	
	


%\newpage
%\addcontentsline{toc}{section}{Glossar}
%\printglossary[type=main]

\newpage
\addcontentsline{toc}{section}{Abk�rzungsverzeichnis}
\section*{Abk�rzungsverzeichnis}
\begin{acronym}[BA]
\acro{patdb}[SHIP - DB]{SHIP - Patientendashboard}
\acro{rapp}[React - App]{React - Applikation}
\acro{rmod}[Modul]{React - Modul}
\acroplural{rmod}[Moduls]{React - Moduls}
\acro{rcom}[Komponente]{React - Komponente}
\acroplural{rcom}[Komponenten]{React - Komponenten}
\end{acronym}

\newpage
%\addcontentsline{toc}{section}{Abbildungsverzeichnis}
\listoffigures

\newpage
\listoftables

\newpage
\addcontentsline{toc}{section}{Listings}
\lstlistoflistings

\newpage
\addcontentsline{toc}{section}{Literatur}
%\bibliography{literatur}
\printbibliography

\newpage
\pagenumbering{gobble}
%\addcontentsline{toc}{section}{Eigenst�ndigkeitserkl�rungen}
\section*{Versicherung der Eigenst�ndigkeit}
Hiermit versichere ich, dass
ich die vorliegende Arbeit selbst�ndig angefertigt und mich keiner fremden Hilfe bedient sowie keine
anderen als die angegebenen Quellen und Hilfsmittel benutzt habe. Alle Stellen, die w�rtlich oder
sinngem�� ver�ffentlichten oder nicht ver�ffentlichten Schriften und anderen Quellen entnommen sind,
habe ich als solche kenntlich gemacht. Diese Arbeit hat in gleicher oder �hnlicher Form noch keiner
Pr�fungsbeh�rde vorgelegen.

\signature{Johannes Lang \\ \textnormal{\textit{Matr. 7217450}} \\
	Henning M�ller \\ \textnormal{\textit{Matr. 7105852}} \\
	Wladislaw Jerokin \\ \textnormal{\textit{Matr. 7205290}}
	}{Dortmund, den \today}

\end{document}