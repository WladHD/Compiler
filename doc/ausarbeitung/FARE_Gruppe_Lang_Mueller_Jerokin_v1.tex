\documentclass[a4paper,11pt,numbers=noenddot]{article}

\usepackage[main=ngerman, english]{babel}
%\usepackage{natbib}
%\bibliographystyle{plainnat}
%\setcitestyle{square,aysep={},yysep={;}}
%\setcitestyle{authoryear,square,semicolon} 
%\setcitestyle{numbers,square}

\usepackage[latin1]{inputenc}
\usepackage[babel, german=quotes]{csquotes} % einfache Handhabung von quotations
\usepackage[backend=bibtex8,style=authoryear]{biblatex} %biblatex mit biber laden
\ExecuteBibliographyOptions{
	sorting=nyt, %Sortierung Autor, Titel, Jahr
	bibwarn=true, %Probleme mit den Daten, die Backend betreffen anzeigen
	isbn=false, %keine isbn anzeigen
	url=true, %keine url anzeigen
	maxcitenames=2,
	maxbibnames=99
}
\renewcommand*{\nameyeardelim}{\addcomma\space}
\addbibresource{res/literaturBibLatex.bib} %Bibliographiedateien laden

\setlength{\parindent}{1em}
\linespread{1.2}
\usepackage[T1]{fontenc}
\usepackage{hyperref}

\hypersetup{
	colorlinks=true,
	linkcolor=black,
	filecolor=black,      
	urlcolor=black,
	citecolor=black,
	pdftitle={FARE Compiler} % PDFTITLE
}


\usepackage{geometry}
\usepackage{textpos} 
\usepackage[table]{xcolor}
\definecolor{lightgray}{gray}{0.9}
\usepackage{graphicx}
\usepackage{float}
\usepackage{csquotes}
\usepackage{ifthen}
\usepackage{booktabs}
\usepackage{tabu}
\usepackage[inline]{enumitem}
\usepackage{enumitem}
\usepackage{calc}
\usepackage{pifont}
\usepackage{colortbl}
\usepackage{longtable}
\usepackage{wrapfig}
\usepackage{listings}
\usepackage{subfig}
\usepackage{ragged2e}
\usepackage{xcolor}

\newcommand\signature[2]{% Name; Department
	\noindent\begin{minipage}{6.5cm}
		\noindent\vspace{3cm}\par
		\noindent\rule{6.5cm}{1pt}\par
		\noindent\textbf{#1}\par
		\noindent#2%
\end{minipage}}

\geometry{a4paper, left=45mm, right=15mm, top=30mm, bottom=25mm, headsep=1cm, headheight=0cm}
\definecolor{accent}{rgb}{0.41, 0.6, 0.76}

\usepackage{fancyhdr}
\usepackage[nottoc]{tocbibind}

\renewcommand{\headrulewidth}{0pt}
\renewcommand{\footrulewidth}{0pt}

\fancyhf{}
\fancyhead[C]{\Roman{page}}
\pagestyle{fancy}

\fancypagestyle{fancy}{}

\definecolor{lightgray}{rgb}{.9,.9,.9}
\definecolor{darkgray}{rgb}{.4,.4,.4}
\definecolor{purple}{rgb}{0.65, 0.12, 0.82}

\lstdefinelanguage{JavaScript}{
	keywords={typeof, new, true, false, catch, function, return, null, catch, switch, var, if, in, while, do, else, case, break},
	keywordstyle=\color{blue}\bfseries,
	ndkeywords={class, export, boolean, throw, implements, import, this},
	ndkeywordstyle=\color{darkgray}\bfseries,
	identifierstyle=\color{black},
	sensitive=false,
	comment=[l]{//},
	morecomment=[s]{/*}{*/},
	commentstyle=\color{purple}\ttfamily,
	stringstyle=\color{black}\ttfamily,
	morestring=[b]',
	morestring=[b]"
}

\lstset{
	language=JavaScript,
	backgroundcolor=\color{lightgray},
	extendedchars=true,
	basicstyle=\footnotesize\ttfamily,
	showstringspaces=false,
	showspaces=false,
	numbers=left,
	numberstyle=\footnotesize,
	numbersep=9pt,
	tabsize=2,
	breaklines=true,
	showtabs=false,
	captionpos=b
}

\newenvironment{changemargin}[2]{%
	\begin{list}{}{%
			\setlength{\topsep}{0pt}%
			\setlength{\leftmargin}{#1}%
			\setlength{\rightmargin}{#2}%
			\setlength{\listparindent}{\parindent}%
			\setlength{\itemindent}{\parindent}%
			\setlength{\parsep}{\parskip}%
		}%
		\item[]}{\end{list}}

\lstdefinelanguage{none}{
	identifierstyle=
}

\colorlet{punct}{red!60!black}
\definecolor{background}{HTML}{EEEEEE}
\definecolor{delim}{RGB}{20,105,176}
\colorlet{numb}{magenta!60!black}

\lstdefinelanguage{json}{
	basicstyle=\footnotesize\ttfamily,
	numbers=left,
	numberstyle=\footnotesize,
	stepnumber=1,
	numbersep=8pt,
	showstringspaces=false,
	breaklines=true,
	frame=lines,
	backgroundcolor=\color{background},
	literate=
	*{0}{{{\color{numb}0}}}{1}
	{1}{{{\color{numb}1}}}{1}
	{2}{{{\color{numb}2}}}{1}
	{3}{{{\color{numb}3}}}{1}
	{4}{{{\color{numb}4}}}{1}
	{5}{{{\color{numb}5}}}{1}
	{6}{{{\color{numb}6}}}{1}
	{7}{{{\color{numb}7}}}{1}
	{8}{{{\color{numb}8}}}{1}
	{9}{{{\color{numb}9}}}{1}
	{:}{{{\color{punct}{:}}}}{1}
	{,}{{{\color{punct}{,}}}}{1}
	{\{}{{{\color{delim}{\{}}}}{1}
	{\}}{{{\color{delim}{\}}}}}{1}
	{[}{{{\color{delim}{[}}}}{1}
	{]}{{{\color{delim}{]}}}}{1},
}


\usepackage[printonlyused]{acronym}

\makeatletter
\renewcommand*\AC@acs[1]{%
	\expandafter\AC@get\csname fn@#1\endcsname\@firstoftwo{#1}}
\makeatother

%biblatex patch, working now
%\linespread{1.25}
%\makeatletter
%\def\blx@err@patch#1{}
%\makeatother

\usepackage{setspace}
\usepackage[automake]{glossaries}
\glsdisablehyper

\newglossary[nlg]{nu}{not}{ntn}{Not Used}

\newglossaryentry{Widget}
{
	type=nu,
	name=Widget,
	description={Ein Widget ist innerhalb des Dashboards ein Element, welches Informationen anzeigt und interaktiv verwendet werden kann},
	plural=Widgets,
	%nonumberlist
}

\makeglossaries

\colorlet{punct}{red!60!black}
\definecolor{background}{HTML}{EEEEEE}
\definecolor{delim}{RGB}{20,105,176}
\colorlet{numb}{magenta!60!black}
\definecolor{accent}{rgb}{0.41, 0.6, 0.76}

\begin{document}
	\begin{titlepage}
		\begin{textblock}{6.5}(-2.5,-3)
			\begin{color}{accent}
				\rule{6.6cm}{33cm}    
			\end{color}
		\end{textblock}
		\begin{textblock}{6.5}(-2,-1)
			{\large \textsf{Ausarbeitung}}
		\end{textblock}
		
		\doublespacing
		\begin{textblock}{8.2}(3.1,1)
			{
				\noindent \LARGE 
				\textsf{\textbf{Entwicklung eines Compilers f�r die Sprache FARE zur Zielsprache Java \\[0.3cm]
			} }}
		\end{textblock}
		\onehalfspacing
		
		\begin{textblock}{8.2}(3.1,3.5)
			{\noindent \large
				\textsf{\textbf{Development of a compiler for the language FARE\\ to the target language Java  \\[0.3cm]
			} }}
		\end{textblock}
		
		\begin{textblock}{6}(3.1,5.5)
			\noindent
			\textsf{An der Fachhochschule Dortmund\\
				im Fachbereich Informatik\\
				Studiengang Medizinische Informatik Master\\
				im Modul Formale Sprachen und Compilerbau\\
				erstellte Ausarbeitung eines FARE-Compilers
			}
		\end{textblock}
		
		
		
		
		\begin{textblock}{6}(-2,7.5)\noindent
			\textsf{von \\Johannes Lang \\
				Matr.-Nr. 7217450 \\ [0.2cm]
				Henning M�ller \\
				Matr.-Nr. 7105852 \\ [0.2cm]
				Wladislaw Jerokin \\
				Matr.-Nr. 7205290 \\ [1cm]
				Betreuung durch: \\
				Prof. Dr. Robert Rettinger \\ [1cm]
				%In enger Zusammenarbeit mit: \\
				%Dr. Georg Lodde \\ [1cm]
				Dortmund, \today
			}
		\end{textblock}
		
		%	\begin{textblock}{6.5}(-1,10.8)
			%		\noindent
			%			\textsf{An der Fachhochschule Dortmund\\
				%			im Fachbereich Informatik\\
				%			Studiengang Medizinische Informatik\\
				%			erstellte Projektarbeit f�r das \\
				%			Modul Wissenschaftliches Arbeiten
				%		}
			%	\end{textblock}
		
	\end{titlepage}
	
	\onehalfspacing
	\setlength\arrayrulewidth{1.1pt}
	\newpage
	\tableofcontents
	
	\newpage
	\setcounter{page}{1}
	\fancyhf{}
	\fancyhead[C]{\thepage}
	\pagestyle{fancy}
	\fancypagestyle{fancy}{}
	
	\section{Einleitung}
	Grundlegend definiert sich ein Compiler als Programm, welches einen gegebenen Quellcode zu Maschinencode, Bytecode oder einer anderen Programmiersprache �bersetzen kann \parencite[vgl.][]{RobertSheldon.2023}.
	Die Entwicklung eines solchen Compilers ist eine komplexe Aufgabe, die aus mehreren Teilgebieten besteht.
	In dieser Ausarbeitung werden folgende Teilgebiete behandelt:
	
	\begin{longtable}{|c|c|}
		\toprule{}
		Num & Name \\ \midrule  
		
		1 & Lexikalische Analyse \\
		
		2 & Syntaxanalyse  \\
		
		3 & Semantische Analyse  \\
		
		4 & Fehlerbehandlung  \\
		
		5 & Codeerzeugung  \\ \midrule
		
		\caption{Behandelte f�nf Teilbereiche des Compilerbaus}
		\label{table:bereiche}
	\end{longtable}
	
	\noindent
	Die in dieser Ausarbeitung behandelte Aufgabe besteht darin, die Bereiche in Tabelle \ref{table:bereiche} f�r eine Scriptsprache f�r den Umgang mit Dateien und Pfaden zu entwickeln.
	Folgend werden die herausgearbeiteten Token, Grammatik und Semantik in jeweils eigenen Kapiteln beschrieben.
	Der Compiler wird in Java geschrieben und benutzt die Bibliotheken JavaCC und die JavaCC-interne JJTree f�r Bereiche 1 und 2 in Tabelle \ref{table:bereiche}.
	
	
	\section{Token}
	\begin{lstlisting}[caption={Definierte JavaCC Token}, label=lst:fhir:impl:VerlaufsdokuAllgemein]
		Observation[...]
		&_sort=-date
		&status:not=entered-in-error
		&identifier=[...]HIS/Cerner/Medico/Observation/ClinicalImpression|
		&subject=<Patient>
		_include=Observation:encounter
	\end{lstlisting}
	
	
	
	
	
	
	


%\newpage
%\addcontentsline{toc}{section}{Glossar}
%\printglossary[type=main]

\newpage
\addcontentsline{toc}{section}{Abk�rzungsverzeichnis}
\section*{Abk�rzungsverzeichnis}
\begin{acronym}[BA]
\acro{patdb}[SHIP - DB]{SHIP - Patientendashboard}
\acro{rapp}[React - App]{React - Applikation}
\acro{rmod}[Modul]{React - Modul}
\acroplural{rmod}[Moduls]{React - Moduls}
\acro{rcom}[Komponente]{React - Komponente}
\acroplural{rcom}[Komponenten]{React - Komponenten}
\end{acronym}

\newpage
%\addcontentsline{toc}{section}{Abbildungsverzeichnis}
\listoffigures

\newpage
\listoftables

\newpage
\addcontentsline{toc}{section}{Listings}
\lstlistoflistings

\newpage
\addcontentsline{toc}{section}{Literatur}
%\bibliography{literatur}
\printbibliography

\newpage
\pagenumbering{gobble}
%\addcontentsline{toc}{section}{Eigenst�ndigkeitserkl�rungen}
\section*{Versicherung der Eigenst�ndigkeit}
Hiermit versichere ich, dass
ich die vorliegende Arbeit selbst�ndig angefertigt und mich keiner fremden Hilfe bedient sowie keine
anderen als die angegebenen Quellen und Hilfsmittel benutzt habe. Alle Stellen, die w�rtlich oder
sinngem�� ver�ffentlichten oder nicht ver�ffentlichten Schriften und anderen Quellen entnommen sind,
habe ich als solche kenntlich gemacht. Diese Arbeit hat in gleicher oder �hnlicher Form noch keiner
Pr�fungsbeh�rde vorgelegen.

\signature{Johannes Lang \\ \textnormal{\textit{Matr. 7217450}} \\
	Henning M�ller \\ \textnormal{\textit{Matr. 7105852}} \\
	Wladislaw Jerokin \\ \textnormal{\textit{Matr. 7205290}}
	}{Dortmund, den \today}

\end{document}